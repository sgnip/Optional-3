% -*- root: SEPRO2451_Opt-assig-3_IMP_GJM_VDJS.tex -*-
\section{Problem 2}

The company ``Compan\'ia de Comunicaciones Terabit'' is planning to build a communication software for which it has been estimated a size of 200 function points. The programming language intended to be used is Java, being one adjusted function point equal to 53 source code lines. The scaling factors applicable are the ones achieved in the problem 1. Please, choose the appropriate COCOMO II submodel and determine the effort, duration and human resources needed to develop the project as well as the productivity related to the project.

\seprule

We have chosen the \textbf{Post-Architecture} submodel: the framework for the development has been established, a size estimation has been performed based on function points, and the tools and personnel are ready.

We used \textit{USC-COCOMO} because \textit{CO-STAR} didn't allow us to estimate projects bigger than 5 KLOC. 

Introducing this values in \textit{USC-COCOMO II} gives us a scale factor of \textbf{23.65} (3.72+5.07+4.24+4.38+6.24). Then we introduce its 200 Unadjusted Function Points indicated in the statement and its equivalence with lines of code given by the statement (53 LoC/fp).

This are the values given to multipliers:

\begin{table}[hbtp]
\centering
\begin{tabular}{|c|c|l|}
\hline
\textbf{Multiplier} & \textbf{Value} & \textbf{Justification} \\ \hline
\textbf{DATA} & high & The P/D coefficient is computed as $\frac{DB size}{SLoC} = \frac{9*(2^{20}}{10600} = 890$\\ \hline
\textbf{CPLX} & high & We think that \textit{distributed processing supported by middleware} needs data restructuring\\ & & and simple triggers (according to our knowledge acquired in other courses from college). \\ \hline
\textbf{TIME} & high & $\approx$ 70\%.  \\ \hline
\textbf{STOR} & nominal & $\leq$ 50 \%. \\ \hline
\textbf{ACAP} & high & Given in statement. \\ \hline
\textbf{APEX} & high & 3 years of experience. \\ \hline
\textbf{LTEX} & high & 3 years of experience. \\ \hline
\textbf{TOOL} & low & CASE tool. \\ \hline
\textbf{SCED} & very low & Estimating without SCED multiplier gave us 12.5 months, and $\frac{9}{12.5}=72\% \leq 75\%$. \\\hline
\end{tabular}
\caption{COCOMO multipliers.}
\label{tblMultipliersProblem2}
\end{table}

RUSE, DOCU, PCON, PVOL, PLEX, SITE multipliers are not mentioned in statement, so they have been given nominal value.


Using this multipliers, we obtain this estimation USC-COCOMO-II:

\begin{table}[hbtp]
\begin{tabular}{|c||c|c|c|c|c|c|}
\hline
\textbf{Estimated} & \textbf{Effort (person-month)} & \textbf{Sched (months)} & \textbf{PROD} & \textbf{COST} & \textbf{INST} & \textbf{Staff} \\ \hline
\textbf{Optimistic} & 34.1 & 11.6 & 311.4 & 27246.22 & 2.6 & 2.9 \\
\textbf{Most Likely} & 42.6& 12.5 & 249 & 34057.78 & 3.2 & 3.4 \\
\textbf{Pessimistic} & 53.2 & 13.5 &199.2 & 42572.23 & 4.0 & 3.9\\ \hline
\end{tabular}
\label{tbl_Problem2}
\caption{COCOMO II estimations.}
\end{table}

We summarize all the overall estimations of the project in Table \ref{tbl_Problem1_ALL}, using Most Likely estimation.

\begin{table}[hbtp]
\begin{tabular}{|c|c|} \hline
\textbf{Project Name} & {Communication for Terabit} \\ \hline
\textbf{Total Lines of Code (SLOC)} & {10600} \\ \hline
\textbf{Hours/PM} & {152.00 h/PM} \\ \hline
\textbf{Total Effort} & {42.6 Person-Month} \\ \hline
\textbf{Estimated Duration} & {12.5 Months} \\ \hline
\textbf{Productivity} & {249 SLOC/PM} \\ \hline
\textbf{Human Resources} & We need more information to know this \\ \hline
\end{tabular}
\label{tbl_Problem2_ALL}
\caption{Overall COCOMO II project estimations.}
\end{table}
