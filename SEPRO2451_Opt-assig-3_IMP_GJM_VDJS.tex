\documentclass{article}

\usepackage{fancyhdr} % Cabeceras de página
\usepackage{lastpage} % Módulo para añadir una referencia a la última página
\usepackage{titling} % No tengo claro para qué es esto
\usepackage[left=3cm,right=3cm,top=3cm,bottom=2cm]{geometry} % Márgenes
\usepackage{xspace}
\usepackage{graphicx}
\usepackage{tikz}
\usepackage{float}
\usepackage{eurosym}

\title{Optional Assignment 3}
\date{\today}
\author{V\'ictor de Juan Sanz \and Iv\'an M\'arquez Pardo\and Guillermo Juli\'an Moreno}

\fancyhf{}
\fancypagestyle{plain}{%
	\lhead{\small \itshape \thetitle\, -\, \thedate\, -\, SEPRO}
	\rhead{Group 3 - VdJS, IMP, GJM}
	\cfoot{\thepage\ of \pageref{LastPage}}
	\rfoot{}
}


\newcommand{\seprule}{{\color{gray} \noindent \hspace{40pt} \hrulefill \hspace{40pt} \vspace{13pt}}}

\begin{document}

\maketitle

% -*- root: /SEPRO2451_Opt-assig-3_IMP_GJM_VDJS.tex -*-
\section{Problem 1}

\textbf{The insurance company “El Casta\~nazo” is planning to develop a new software for their insurance policies management for which it has been estimated a size of 184 function points.
The programming environment to be used is Visual Age 2.0. Choose the appropriate COCOMO II submodel and determine the effort, duration and human resources needed to develop the project as well as the productivity related to the project, considering the following scaling factors:}

\begin{itemize}
\item The whole code to be developed for this project is new.
\item The organization is placed in level 1 of its capacity maturity, taking into account that it fulfils the key processes areas (KPAs) of requirements
management and configuration management.
\item This is the first time that Visual Age 2.0 programming environment is to be used.
\item Several applications for the insurance policy management have been developed.
\item The project team is formed by six people, three of them have just been hired.
\item The fact that the software fulfils the preassigned requirements of the software to be developed is seen as a very high need.
\item The scaling factor RESL will not be taken into account for the estimations to be carried out.
\end{itemize}


\seprule

First of all, we have to decide which COCOMO II submodel we are going to apply taking into account the information we have and the needs of the client, the insurance company 'El Casta\~nazo'.

The submodel applied to the requested new software is going to be \emph{Anticipated Development}, as we haven't started its development yet and we are starting from scratch (not using preexisting components), but we are given its size estimation in Unadjusted Function Points: 184 FP.

Next thing to do is giving values to the Scale Factors:

\begin{itemize}
\item \textbf{Precedentness:} On the one hand, this is the first time that Visual Age 2.0 programming environment is to be used (Very Low precedentedness). On the other hand, several applications for the insurance policy management have been developed (Very High precedentedness). Therefore, we consider that a nominal precedentedness might conveniently reflex this situation. \textbf{Value: 3}.
\item \textbf{Development flexibility:} The fact that the software fulfils the preassigned requirements of the software to be developed is seen as a very high need. It seems that the flexibility is going to be very low, as the pre-established requirements have a high priority. \textbf{Value: 5}.
\item \textbf{Architecture/risk resolution:} it will not be taken into account for the estimations we are going to carry out. Therefore, its value will be nominal as we don't have more information about it. \textbf{Value: 3}.
\item \textbf{Team cohesion:} the project team is formed by six people, three of them have just been hired. Half of the team has never worked with the other half and three people don't really know yet how things are done in the company. Therefore, there will be some difficult interactions at first. \textbf{Value: 4} .
\item \textbf{Process maturity:} The organization is placed in level 1 of its capacity maturity, taking into account that it fulfils the key processes areas (KPAs) of requirements management and configuration management. We will suppose that the maturity is Level 1 (upper half) out of 5, which results in a low maturity.\textbf{Value: 4}.
\end{itemize}

Introducing this values in \textit{USC-COCOMO II} gives us a scale factor of \textbf{23.65} ($3.72+5.07+4.24+4.38+6.24$). Then we introduce its 184 Unadjusted Function Points indicated in the statement. VisualAge 2.0 is a software development environment for multiple languages, and after some research on the Internet, it seems version 2.0 is quite related to JAVA. As we don't really know the language in which the software will be coded and JAVA is an Object-Oriented programming language, we have chosen Object-Oriented Default Language and set the corresponding option for our project.

With this information, the estimations obtained are the ones shown in Table \ref{tbl_Problem1}, and we summarize all the overall estimations of the project in Table \ref{tbl_Problem1_ALL}.

\begin{table}
\centering
\begin{tabular}{c|c|c|c|c|c|c}
\textbf{Estimated} & \textbf{Effort (person-month)} & \textbf{Sched (months)} & \textbf{PROD} & \textbf{COST} & \textbf{INST} & \textbf{Staff} \\ \hline
\textbf{Optimistic} & 13.4 & 8.6 & 397.2 & 0.00 & 0.0 & 1.6 \\
\textbf{Most Likely} & 20.0 & 9.8 & 266.1 & 0.00 & 0.0 & 2.0 \\
\textbf{Pessimistic} & 30.1 & 11.2 & 177.4 & 0.00 & 0.0 & 2.7 \\ \hline
\end{tabular}
\label{tbl_Problem1}
\caption{COCOMO II estimations.}
\end{table}


\begin{table}
\centering
\begin{tabular}{|c|c|} \hline
\textbf{Project Name} & \emph{Castaniazo} \\ \hline
\textbf{Total Lines of Code (SLOC)} & \emph{5336} \\ \hline
\textbf{Hours/PM} & \emph{152.00 h/PM} \\ \hline
\textbf{Total Effort} & \emph{20.049 Person-Month} \\ \hline
\textbf{Estimated Duration} & \emph{9.8 Months} \\ \hline
\textbf{Productivity} & \emph{266.1 SLOC/PM} \\ \hline
\textbf{Human Resources} & \emph{2.0 persons\footnotemark} \\ \hline
\end{tabular}
\label{tbl_Problem1_ALL}
\caption{Overall COCOMO II project estimations.}
\end{table}

\footnotetext{$\sim$2 persons in Optimistic and Most Likely Estimations, $\sim$3 persons in the Pessimistic Estimation.}

We don't have more information about Effort Multipliers nor personnel salaries, so we can't get more estimations for our project.


\newpage
\section{Problem 2}

The company ``Compan\'ia de Comunicaciones Terabit'' is planning to build a communication software for which it has been estimated a size of 200 function points. The programming language intended to be used is Java, being one adjusted function point equal to 53 source code lines. The scaling factors applicable are the ones achieved in the problem 1. Please, choose the appropriate COCOMO II submodel and determine the effort, duration and human resources needed to develop the project as well as the productivity related to the project.

\seprule

We have chosen the \textbf{Post-Architecture} submodel: the framework for the development has been established, a size estimation has been performed based on function points, and the tools and personnel are ready.

We used \textit{USC-COCOMO} because \textit{CO-STAR} didn't allow us to estimate projects bigger than 5 KLOC. 

Introducing this values in \textit{USC-COCOMO II} gives us a scale factor of \textbf{23.65} (3.72+5.07+4.24+4.38+6.24). Then we introduce its 200 Unadjusted Function Points indicated in the statement and its equivalence with lines of code given by the statement (53 LoC/fp).

This are the values given to multipliers:

\begin{table}[hbtp]
\centering
\begin{tabular}{l|l|l}
Multiplier & Value & Justification \\ \hline
DATA & high & The P/D coefficient is computed as $\frac{DB size}{SLoC} = \frac{9*(2^{20}}{10600} = 890$\\ \hline
CPLX & high & We think that \textit{distributed processing supported by middleware} needs data restructuring and simple triggers (according to our knowledge acquired in other courses from college). \\ \hline
TIME & high & $\approx$ 70\%.  \\ \hline
STOR & nominal & $\leq$ 50 \%. \\ \hline
ACAP & high & Given in statement. \\ \hline
APEX & high & 3 years of experience. \\ \hline
LTEX & high & 3 years of experience. \\ \hline
TOOL & low & CASE tool. \\ \hline
SCED & nominal & Because no stretch-out nor compression are needed. \\ \hline
\end{tabular}
\caption{COCOMO multipliers.}
\label{tblMultipliersProblem2}
\end{table}

RUSE,DOCU, PCON, PVOL, PLEX, SITE multipliers are not mentioned in statement, so they have been given nominal value.



Using this multipliers, we obtain this estimation USC-COCOMO-II:

\begin{table}
\begin{tabular}{c|c|c|c|c|c|c}
\textbf{Estimated} & \textbf{Effort (person-month)} & \textbf{Sched (months)} & \textbf{PROD} & \textbf{COST} & \textbf{INST} & \textbf{Staff} \\ \hline
\textbf{Optimistic} & 34.1 & 11.6 & 311.4 & 27246.22 & 2.6 & 2.9 \\
\textbf{Most Likely} & 42.6& 12.5 & 249 & 34057.78 & 3.2 & 3.4 \\
\textbf{Pessimistic} & 53.2 & 13.5 &199.2 & 42572.23 & 4.0 & 3.9\\ \hline
\end{tabular}
\label{tbl_Problem2}
\caption{COCOMO II estimations.}
\end{table}

We summarize all the overall estimations of the project in Table \ref{tbl_Problem1_ALL}, using Most Likely estimation.

\begin{table}
\begin{tabular}{|c|c|} \hline
\textbf{Project Name} & \emph{Communication for Terabit} \\ \hline
\textbf{Total Lines of Code (SLOC)} & \emph{10600} \\ \hline
\textbf{Hours/PM} & \emph{152.00 h/PM} \\ \hline
\textbf{Total Effort} & \emph{42.6 Person-Month} \\ \hline
\textbf{Estimated Duration} & \emph{12.5 Months} \\ \hline
\textbf{Productivity} & \emph{249 SLOC/PM} \\ \hline
\textbf{Human Resources} & We need more information to know this \\ \hline
\end{tabular}
\label{tbl_Problem1_ALL}
\caption{Overall COCOMO II project estimations.}
\end{table}

\newpage
\appendix

% -*- root: /SEPRO2451_Opt-assig-3_IMP_GJM_VDJS.tex -*-
\section{Working Time/Time Consumed}
In this section we gather the number of hours that each member of the team has spent in order to complete this assignment.

\begin{table}
\begin{tabular}{|c|c|p{8cm}|} \hline
\textbf{Person} & \textbf{Total hours} & \textbf{Activities} \\ \hline
Vi\'ctor de Juan Sanz & TODO & TODO \\ \hline
Guillermo Julia\'n Moreno & TODO & TODO \\ \hline
Iva\'n Márquez Pardo & ~4 hours & 
\begin{itemize}
\item Read recommended documentation (COCOMO II PDF and slides): \emph{1,5 hours (1h + 0.5h)}.
\item Problem analysis and writing down some useful notes: \emph{0,5 hours}.
\item Work on the problem (solving) + Writing down answers, reasoning and COCOMO II results: \emph{~2 hours}.
\end{itemize} \\ \hline
\end{tabular}
\end{table}

\end{document}
