\documentclass{article}

\usepackage{fancyhdr} % Cabeceras de página
\usepackage{lastpage} % Módulo para añadir una referencia a la última página
\usepackage{titling} % No tengo claro para qué es esto
\usepackage[left=3cm,right=3cm,top=3cm,bottom=2cm]{geometry} % Márgenes
\usepackage{xspace}
\usepackage{graphicx}
\usepackage{tikz}
\usepackage{float}
\usepackage{eurosym}

\title{Optional Assignment 3}
\date{\today}
\author{V\'ictor de Juan Sanz \and Iv\'an M\'arquez Pardo\and Guillermo Juli\'an Moreno}

\fancyhf{}
\fancypagestyle{plain}{%
	\lhead{\small \itshape \thetitle\, -\, \thedate\, -\, SEPRO}
	\rhead{Group 3 - VdJS, IMP, GJM}
	\cfoot{\thepage\ of \pageref{LastPage}}
	\rfoot{}
}


\newcommand{\seprule}{{\color{gray} \noindent \hspace{40pt} \hrulefill \hspace{40pt} \vspace{13pt}}}

\begin{document}

\maketitle

\section{Problem 1}

The insurance company “El Casta\~nazo” is planning to develop a new software for their
insurance policies management for which it has been estimated a size of 184 function
points. The programming environment to be used is Visual Age 2.0. Choose the appropriate method to develop the project as well as the productivity related to the project.

\seprule

We have chosen the \textbf{Post-Architecture} submodel: the framework for the development has been established, a size estimation has been performed based on function points, and the tools and personnel are ready.

For the COCOMO parameters, we're choosing a very high precedentedness \textit{PREC} (the company has already developed several similar applications), very low flexibility \textit{FLEX}, low team cohesion \textit{TEAM} given that half of the team has been hired recently and low process maturity \textit{PMAT}.

The results are shown in table \ref{tblProblem1}.

% TODO: Fill this table

\begin{table}[hbtp]
\centering
\begin{tabular}{l|l}
Effort & X person-months \\ \hline
Duration & X months \\ \hline
Human resources & ? \\ \hline
Productivity & ?
\end{tabular}
\caption{COCOMO estimations.}
\label{tblProblem1}
\end{table}

\section{Problem 2}

The company ``Compan\'ia de Comunicaciones Terabit'' is planning to build a communication software for which it has been estimated a size of 200 function points. The programming language intended to be used is Java, being one adjusted function point equal to 53 source code lines. The scaling factors applicable are the ones achieved in the problem 1. Please, choose the appropriate COCOMO II submodel and determine the effort, duration and human resources needed to develop the project as well as the productivity related to the project.

\seprule

We have chosen the \textbf{Post-Architecture} submodel: the framework for the development has been established, a size estimation has been performed based on function points, and the tools and personnel are ready.

% TODO: Fill this table

\begin{table}[hbtp]
\centering
\begin{tabular}{l|l}
Effort & X person-months \\ \hline
Duration & X months \\ \hline
Human resources & ? \\ \hline
Productivity & ? \\ \hline
Development cost & Y \euro
\end{tabular}
\caption{COCOMO estimations.}
\label{tblProblem1}
\end{table}

\appendix

\section{Number of hours spent in this activity}

All of three team members have spent 1 hour reading and understanding the COCOMO manual.

\end{document}
